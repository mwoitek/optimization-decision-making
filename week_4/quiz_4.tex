% Created 2024-07-01 Mon 20:47
% Intended LaTeX compiler: pdflatex
\documentclass[11pt]{article}
\usepackage[utf8]{inputenc}
\usepackage[T1]{fontenc}
\usepackage{graphicx}
\usepackage{longtable}
\usepackage{wrapfig}
\usepackage{rotating}
\usepackage[normalem]{ulem}
\usepackage{amsmath}
\usepackage{amssymb}
\usepackage{capt-of}
\usepackage{hyperref}
\usepackage[a4paper,left=1cm,right=1cm,top=1cm,bottom=1cm]{geometry}
\usepackage[american, english]{babel}
\usepackage{enumitem}
\usepackage{float}
\usepackage[sc]{mathpazo}
\linespread{1.05}
\renewcommand{\labelitemi}{$\rhd$}
\setlength\parindent{0pt}
\setlist[itemize]{leftmargin=*}
\setlist{nosep}
\date{}
\title{Quiz: A Medical Device Example}
\hypersetup{
 pdfauthor={Marcio Woitek},
 pdftitle={Quiz: A Medical Device Example},
 pdfkeywords={},
 pdfsubject={},
 pdfcreator={Emacs 29.4 (Org mode 9.8)}, 
 pdflang={English}}
\begin{document}

\thispagestyle{empty}
\pagestyle{empty}
\section*{Problem 1}
\label{sec:orge3a5f91}

\textbf{Answer:} 3\\

There are three decision variables, \(x_1\), \(x_2\) and \(x_3\). The
variable \(x_i\) corresponds to the number of pig valves purchased from the
\(i\)-th supplier.
\section*{Problem 2}
\label{sec:orge7857ab}

\textbf{Answer:} \(0.4x_1+0.3x_2+0.2x_3\geq 500\)\\

In total, \(x_1\) valves are purchased from supplier 1. In this case, the
number of large valves is 40\% of the total. Then \(0.4x_1\) large valves are
purchased from supplier 1. A similar argument allows us to conclude that
\begin{itemize}
\item \(0.3x_2\) large valves are purchased from supplier 2, and
\item \(0.2x_3\) large valves are purchased from supplier 3.
\end{itemize}
Therefore, the total number of large valves can be expressed as
\(0.4x_1+0.3x_2+0.2x_3\). At least 500 such valves need to be purchased every
month. Hence the following must hold:
\begin{equation}
0.4x_1+0.3x_2+0.2x_3\geq 500.
\end{equation}
\section*{Problem 3}
\label{sec:org4c5cbda}

\textbf{Answer:} \(0.4x_1+0.35x_2+0.2x_3\geq 300\)\\

In total, \(x_1\) valves are purchased from supplier 1. In this case, the
number of medium valves is 40\% of the total. Then \(0.4x_1\) medium valves are
purchased from supplier 1. A similar argument allows us to conclude that
\begin{itemize}
\item \(0.35x_2\) medium valves are purchased from supplier 2, and
\item \(0.2x_3\) medium valves are purchased from supplier 3.
\end{itemize}
Therefore, the total number of medium valves can be expressed as
\(0.4x_1+0.35x_2+0.2x_3\). At least 300 such valves need to be purchased every
month. Hence the following must hold:
\begin{equation}
0.4x_1+0.35x_2+0.2x_3\geq 300.
\end{equation}
\section*{Problem 4}
\label{sec:org897f3c7}

\textbf{Answer:} \(0.2x_1+0.35x_2+0.6x_3\geq 300\)\\

In total, \(x_1\) valves are purchased from supplier 1. In this case, the
number of small valves is 20\% of the total. Then \(0.2x_1\) small valves are
purchased from supplier 1. A similar argument allows us to conclude that
\begin{itemize}
\item \(0.35x_2\) small valves are purchased from supplier 2, and
\item \(0.6x_3\) small valves are purchased from supplier 3.
\end{itemize}
Therefore, the total number of small valves can be expressed as
\(0.2x_1+0.35x_2+0.6x_3\). At least 300 such valves need to be purchased every
month. Hence the following must hold:
\begin{equation}
0.2x_1+0.35x_2+0.6x_3\geq 300.
\end{equation}
\section*{Problem 5}
\label{sec:orgb9bd321}

\textbf{Answer:} Decision variables
\section*{Problem 6}
\label{sec:orgee350ec}

\textbf{Answer:} Objective function coefficients
\section*{Problem 7}
\label{sec:org0bacb2d}

\textbf{Answer:} Constraint function coefficients
\section*{Problem 8}
\label{sec:orgd33200f}

\textbf{Answer:} \texttt{SUMPRODUCT(\$B\$4:\$D\$4,B10:D10)}
\section*{Problem 9}
\label{sec:org752ce50}

\textbf{Answer:} All of the above
\section*{Problem 10}
\label{sec:orgb01e82e}

\textbf{Answer:} \(\$ 6450\)
\end{document}
