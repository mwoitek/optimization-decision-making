% Created 2024-06-29 Sat 23:42
% Intended LaTeX compiler: pdflatex
\documentclass[11pt]{article}
\usepackage[utf8]{inputenc}
\usepackage[T1]{fontenc}
\usepackage{graphicx}
\usepackage{longtable}
\usepackage{wrapfig}
\usepackage{rotating}
\usepackage[normalem]{ulem}
\usepackage{amsmath}
\usepackage{amssymb}
\usepackage{capt-of}
\usepackage{hyperref}
\usepackage[a4paper,left=1cm,right=1cm,top=1cm,bottom=1cm]{geometry}
\usepackage[american, english]{babel}
\usepackage{enumitem}
\usepackage{float}
\usepackage[sc]{mathpazo}
\linespread{1.05}
\renewcommand{\labelitemi}{$\rhd$}
\setlength\parindent{0pt}
\setlist[itemize]{leftmargin=*}
\setlist{nosep}
\date{}
\title{Quiz: Giapetto's Woodcarving Company}
\hypersetup{
 pdfauthor={Marcio Woitek},
 pdftitle={Quiz: Giapetto's Woodcarving Company},
 pdfkeywords={},
 pdfsubject={},
 pdfcreator={Emacs 29.4 (Org mode 9.8)}, 
 pdflang={English}}
\begin{document}

\thispagestyle{empty}
\pagestyle{empty}
\section*{Problem 1}
\label{sec:org6fba4a7}

\textbf{Answer:} 2\\

There are two decision variables, \(x_1\) and \(x_2\). \(x_1\) denotes the
number of toy soldiers produced, and \(x_2\) denotes the number of toy trains
produced.
\section*{Problem 2}
\label{sec:org57f9fff}

\textbf{Answer:} \(\$ 3\)\\

The cost of producing a toy soldier is \(\$ 24\) (materials + labor). Since a
soldier sells for \(\$ 27\), the profit margin is \(\$ 3\).
\section*{Problem 3}
\label{sec:orgdd132d5}

\textbf{Answer:} \(\$ 2\)\\

The cost of producing a toy train is \(\$ 19\) (materials + labor). Since a
train sells for \(\$ 21\), the profit margin is \(\$ 2\).
\section*{Problem 4}
\label{sec:org3f9116a}

\textbf{Answer:}
\begin{equation*}
\max\quad Z=3x_1+2x_2
\end{equation*}
\vspace{0.1cm}

Producing \(x_1\) soldiers generates a profit of \(3x_1\). Similarly,
producing \(x_2\) trains generates a profit of \(2x_2\). Then the total
profit is \(Z=3x_1+2x_2\). The goal is to maximize profit. Therefore, Giapetto
needs to find \(\max Z\).
\section*{Problem 5}
\label{sec:orgc58b71e}

\textbf{Answer:} 3\\

There are three constraints:
\begin{itemize}
\item one that specifies the upper bound for the number of finishing hours;
\item one that specifies the upper bound for the number of carpentry hours;
\item one that specifies the upper bound for the number of toy soldiers.
\end{itemize}
\section*{Problem 6}
\label{sec:orgf9efdbf}

\textbf{Answer:} \(2x_1+x_2\leq 100\)\\

To produce \(x_1\) soldiers, \(2x_1\) hours of finishing labor are required.
Similarly, to produce \(x_2\) trains, \(x_2\) hours of finishing labor are
required. Then the total number of finishing hours is \(2x_1+x_2\). We know
this number is at most 100 hours. Hence:
\begin{equation*}
2x_1+x_2\leq 100.
\end{equation*}
\section*{Problem 7}
\label{sec:org43213f1}

\textbf{Answer:} \(x_1+x_2\leq 80\)\\

To produce \(x_1\) soldiers, \(x_1\) hours of carpentry labor are required.
Similarly, to produce \(x_2\) trains, \(x_2\) hours of carpentry labor are
required. Then the total number of carpentry hours is \(x_1+x_2\). We know
this number is at most 80 hours. Hence:
\begin{equation*}
x_1+x_2\leq 80.
\end{equation*}
\section*{Problem 8}
\label{sec:orgf76509b}

\textbf{Answer:} No\\

Actually, it's not necessary to plot the feasible region. We can answer this
question by using the constraints directly. For \(x_1=40\) and \(x_2=30\),
we have
\begin{align*}
2x_1+x_2&=2\cdot 40+30\\
&=80+30\\
&=110.
\end{align*}
This shows that the constraint \(2x_1+x_2\leq 100\) is not satisfied.
Therefore, the point under consideration is \textbf{not} in the feasible region.
\section*{Problem 9}
\label{sec:orga030307}

\textbf{Answer:} \((x_1=20,x_2=60)\)\\

The plot clearly shows that the solution corresponds to the point \(G\). This
point lies in the intersection between the lines defined by the following
equations:
\begin{align*}
2x_1+x_2&=100,\\
x_1+x_2&=80.
\end{align*}
Using the second equation, we can express \(x_2\) as
\begin{equation*}
x_2=80-x_1.
\end{equation*}
Next, we substitute this result into the first equation, and solve for \(x_1\):
\begin{align*}
2x_1+x_2&=100\\
2x_1+80-x_1&=100\\
x_1+80&=100\\
x_1&=100-80\\
x_1&=20
\end{align*}
Hence:
\begin{equation*}
x_2=80-20=60.
\end{equation*}
Therefore, the optimal solution is given by \((x_1,x_2)=(20,60)\). Notice that
this result is consistent with what is shown in the graph.
\section*{Problem 10}
\label{sec:org431a886}

\textbf{Answer:} \(\$ 180\)\\

According to the graph, this value is \(Z=180\). Just to be sure, let's use
the optimal solution to compute the maximum profit:
\begin{align*}
Z&=3x_1+2x_2\\
&=3\cdot 20+2\cdot 60\\
&=60+120\\
&=180.
\end{align*}
\end{document}
